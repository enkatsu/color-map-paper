% \documentclass[a4paper,twocolumn]{jsarticle}
\documentclass[a4j,twocolumn]{ujarticle} % 'jsarticle' が使えない場合はこちらを利用

% 上記 documentclass のオプションは自由に追加してもよい.

\newcommand{\workname}{緋色の習作}
\newcommand{\worknameen}{A Study in Scarlet}
\newcommand{\colorname}{色彩語}
\newcommand{\colorkanji}{色彩漢字}
\newcommand{\recallcolor}{色イメージ}
\newcommand{\mysection}[1]{\vspace{-23pt}\section{#1}\vspace{-5pt}}
\newcommand{\mysubsection}[1]{\vspace{-19pt}\subsection{#1}\vspace{-5pt}}
\newcommand{\myfigure}[3]{
\begin{figure}[htbp]
  \begin{center}
    \includegraphics[width=7cm,clip]{#1}
    \caption{#2}
    \vspace{-2zh}
    \label{#3}
  \end{center}
\end{figure}
}

\setlength\intextsep{10pt}
\setlength\textfloatsep{2truemm}
\setlength\abovecaptionskip{1truemm}

%%%%%%%%%%%%%%%%%%%%%%%%%%%%%%%%%%%%%%%%%%%%%%%%%%%%%%%%%%%%%%%%%%%%%%%%%%%%%%
%%% ページ設定 (この項目は論文著者は編集しないこと.)

% 芸術科学会学術会議用スタイルパッケージ
\usepackage{artsci-conf-j}

% 開始ページ数設定
\setcounter{page}{1}

% ヘッダスタイル
\markright{\footnotesize{
\textbf{NICOGRAPH 20xx, pp. aaa -- bbb}
}}
\pagestyle{myheadings}

%%%%%%%%%%%%%%%%%%%%%%%%%%%%%%%%%%%%%%%%%%%%%%%%%%%%%%%%%%%%%%%%%%%%%%%%%%%%%%
%%% パッケージ一覧 (必要なパッケージを任意に追加してよい)

\usepackage{amsmath, amssymb}	% AMS-LaTeX
\usepackage[dvipdfmx]{graphicx}	% 「graphics」パッケージに変更してもよい.
\usepackage{float}		% 図表が記述位置から飛ばないためのパッケージ
\usepackage{url}		% URL 表記用パッケージ

%%%%%%%%%%%%%%%%%%%%%%%%%%%%%%%%%%%%%%%%%%%%%%%%%%%%%%%%%%%%%%%%%%%%%%%%%%%%%%
%%% 画像ファイルの include path
\graphicspath{{fig/}} %% グラフィック用


%%%%%%%%%%%%%%%%%%%%%%%%%%%%%%%%%%%%%%%%%%%%%%%%%%%%%%%%%%%%%%%%%%%%%%%%%%%%%%
%%% マクロ一覧 (必要なマクロをこの部分に記述)

\newcommand{\bA}{\mathbf{A}}
\newcommand{\bB}{\mathbf{B}}


%%%%%%%%%%%%%%%%%%%%%%%%%%%%%%%%%%%%%%%%%%%%%%%%%%%%%%%%%%%%%%%%%%%%%%%%%%%%%%
%%% タイトル,著者,所属,概要

% 日本語タイトル
\jtitle{
\workname{}:\colorname{}に用いられる漢字と\\
\colorname{}が表す色の関係性に関する可視化手法の提案と考察
}

% 英語タイトル
\etitle{
\worknameen{}: Proposal and Consideration of Visualization Method of Relationship between Kanji Used for Color Words and Colors Represented by Color Words
}

% 日本語著者
% 所属参照は好みに応じて \dagger などを用いてもよい.
\jauthor{
遠藤勝也\(^{1)}\){\small (正会員)}
% ~~~ % 氏名の間隔はチルダ記号や hspace 等で適宜調節のこと.
% 科学次郎\(^{2)}\){\small (正会員)}
}

% 英語著者
\eauthor{
Katsuya Endoh\(^{1)}\)
% ~~~
% Jiro Kagaku\(^{2)}\)
}

% 日本語所属
\jaffiliation{
1) 株式会社スタジオ・アルカナ
% ~~~
% 2) 芸術科学大学芸術科学部
}

% 英語所属
\eaffiliation{
1) Studio Arcana co.,Ltd.\\
% 2) Department of Art and Science, The University for Art and Science
}

% 連絡先電子メールアドレス(省略可)
% (スパム対策は著者自身の判断によって措置すること.
% このサンプルでは「@」を2バイト文字にすることで対応してある.)
\email{
endkty0509@gmail.com
}

% 日本語概要
\jabstract{
本研究では\colorname{}を漢字に分解することで,
\colorname{}に用いられる漢字と色の関係性を可視化した作品「\workname{}」を制作した.
本稿では,まず制作の背景について述べ,先行研究および関連作品をあげた上で,本作品について説明する.
また,本作品によって得られた,\colorname{}に用いられる漢字と色の関係性について考察し,
今後の展望について述べる.
}

%%%%%%%%%%%%%%%%%%%%%%%%%%%%%%%%%%%%%%%%%%%%%%%%%%%%%%%%%%%%%%%%%%%%%%%%%%%%%%
% ここより論文本体

\begin{document}
\maketitle
\thispagestyle{myheadings}

\mysection{はじめに}

色を表記する方法として,\colorname{}を用いることがある.
たとえばRGB値で(134, 71, 63)と表される色は,
深緋(こきあけ)という\colorname{}で表される.
上記の値から暗く濃い赤色だということは想像できるが,
色を数値で扱う習慣がなければ難しいと考えられる.
しかし,深緋という\colorname{}が,
「深」と「緋」の二つの漢字から構成されていることに注目すると,
日本語話者には「深」という漢字からは暗さや濃さ,
「緋」という漢字からは赤に近い色のような感覚が想起されると考えられる.
そこで,本研究では\colorname{}を漢字に分解することで,
\colorname{}に用いられる漢字と色の関係性を可視化した作品「\workname{}」を制作し,
その関係性について考察する.

\mysection{先行研究および関連作品}

色彩語における漢字の果たす役割として,
木村ら\cite{Kimura1998}は「赤」や「紅」「朱」という漢字は,
比較可能な環境にある時には,
想起させる色の差は大きくなるが,
比較対照する環境下に置かなければ,
きわめて似通ったものとして認識していると述べている.

また,日本の伝統色を可視化した作品として,
NIPPON COLORS\cite{NipponColors}があげられる.
この作品は日本の伝統色を扱った色見本サイトだが,
色見本の表示方法として,各色をマンセルの色立体を使って可視化している.

上記のように,色彩語における漢字の果たす役割に関する研究や,
日本の伝統色を可視化した作品の制作は多く行われているが,
色彩語を構成する漢字の関係性を可視化した作品は少ない.

\vspace{-1zh}

\begin{figure*}[h]
  \begin{center}
    \begin{minipage}{0.3\hsize}
      \begin{center}
        \includegraphics[width=5.7cm]{fig/kanji-color-space.eps}
      \end{center}
      \caption{No.1}
      \label{fig:no1}
    \end{minipage}
    \begin{minipage}{0.3\hsize}
      \begin{center}
        \includegraphics[width=5.7cm]{fig/kanji-color-graph.eps}
      \end{center}
      \caption{No.2}
      \label{fig:no2}
    \end{minipage}
    \begin{minipage}{0.3\hsize}
      \begin{center}
        \includegraphics[width=5.7cm]{fig/kanji-circle-color-graph.eps}
      \end{center}
      \caption{No.3}
      \label{fig:no3}
    \end{minipage}
  \end{center}
\end{figure*}

\vspace{-1zh}

\section{「\workname{}」について}

「\workname{}」はNo.1からNo.3の三つの可視化作品から構成される.
作品には,
2019年6月15日時点でのWikipedia色名一覧のページ\cite{WikiColorName}の情報を元に
構築したデータベースを使用した.
このデータベースには,
404語の\colorname{}と,
\colorname{}に紐づくRGB値で表現された色が登録されている.
また,\colorname{}は298種類の漢字の組み合わせで構成されている.
以下の節でそれぞれの作品について述べる.

\mysubsection{No.1について}

No.1は,
\colorname{}を漢字に分解し,
それぞれの漢字が用いられる色の平均色を計算し,
計算された色を元にHSV空間内に配置した可視化作品である.
図\ref{fig:no1}に動作している様子を示す.

\mysubsection{No.2について}

No.2は,
\colorname{}をノード,
同じ漢字を使用している関係性をエッジとして,
グラフ構造を可視化した作品である.
ノードは力学モデルによって配置されるため,
同じ漢字を使用している\colorname{}のクラスタを観察することが可能である.
また,ノードにマウスオーバーすることで,
ノードが示す\colorname{}が表示される.
図\ref{fig:no2}に動作している様子を示す.

\mysubsection{No.3について}

No.3は,
No.2と同様の関係性を用いてグラフ構造を可視化した作品である.
ノードは色相によって円形に配置されていているため,
色相環上における\colorname{}の関係性を観察することが可能である.
また,No.2と同様のインタラクションにより\colorname{}が表示される.
図\ref{fig:no3}に動作している様子を示す.

\mysection{考察}
\label{consideration}

No.1を観察した結果,
「海」という漢字に注目した時に,
青に近い色を想起させられると考えられるが,
平均色は(119, 90, 58)と茶色に近い色となった.
この理由として,「海」が用いられている\colorname{}は,
6語の\colorname{}のうち,3語が「海老」として用いられていたからだと考えられる.
この結果から,\colorname{}を構成する単語も考慮する必要があると考えられる.

またNo.2の,
「赤」や「紅」,「朱」,「緋」に注目したところ,
「赤」が作っているクラスタは色彩がまばらなように見られたが,
「紅」や「朱」,「緋」が作っているクラスタは,
赤に近い色相の範囲で,わずかに異なる色の系統を表しているように見られた.
よって,これらの漢字が表す色には規則的な差異があると考えられる.

最後にNo.3からは,
茶色周辺は多くのエッジが密集しているが,
青周辺はエッジが分散していることが観察できた,
このことから茶色を表す漢字は少なく,
青を表す漢字は豊富だということが考えられる.

\mysection{おわりに}

本稿では,可視化作品「\workname{}」を制作し,
\colorname{}に用いられる漢字と色の関係性を考察した.
今後の展望として,
本作品から得られた考察を定量的に評価し検証することで,
新たな視点から\colorname{}と色の関係性を可視化することが可能だと考えられる.
また,\colorname{}から想起される色のデータをアンケートから収集することで,
ユーザ特性による想起される色の違いを可視化することが可能だと考えられる.

\vspace{-2.5zh}

%%%%%%%%%%%%%%%%%%%%%%%%%%%%%%%%%%%%%%%%%%%%%%%%%%%%%%%%%%%%%%%%%%%%%%%%%%%%%%
%%% 参考文献

\bibliography{bibtex_samp}
\bibliographystyle{junsrt}

\end{document}
