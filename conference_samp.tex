% \documentclass[a4paper,twocolumn]{jsarticle}
\documentclass[a4j,twocolumn]{ujarticle} % 'jsarticle' が使えない場合はこちらを利用

% 上記 documentclass のオプションは自由に追加してもよい.

\newcommand{\workname}{緋色の習作}
\newcommand{\worknameen}{A study in scarlet}
\newcommand{\colorname}{色彩語}
\newcommand{\colorkanji}{色彩漢字}
\newcommand{\recallcolor}{色イメージ}
\setlength\intextsep{10pt}
\setlength\textfloatsep{2truemm}
\newcommand{\mysection}[1]{\vspace{-20pt}\section{#1}}
\newcommand{\mysubsection}[1]{\vspace{-20pt}\subsection{#1}}

%%%%%%%%%%%%%%%%%%%%%%%%%%%%%%%%%%%%%%%%%%%%%%%%%%%%%%%%%%%%%%%%%%%%%%%%%%%%%%
%%% ページ設定 (この項目は論文著者は編集しないこと.)

% 芸術科学会学術会議用スタイルパッケージ
\usepackage{artsci-conf-j}

% 開始ページ数設定
\setcounter{page}{1}

% ヘッダスタイル
\markright{\footnotesize{
\textbf{NICOGRAPH 20xx, pp. aaa -- bbb}
}}
\pagestyle{myheadings}

%%%%%%%%%%%%%%%%%%%%%%%%%%%%%%%%%%%%%%%%%%%%%%%%%%%%%%%%%%%%%%%%%%%%%%%%%%%%%%
%%% パッケージ一覧 (必要なパッケージを任意に追加してよい)

\usepackage{amsmath, amssymb}	% AMS-LaTeX
\usepackage[dvipdfmx]{graphicx}	% 「graphics」パッケージに変更してもよい.
\usepackage{float}		% 図表が記述位置から飛ばないためのパッケージ
\usepackage{url}		% URL 表記用パッケージ

%%%%%%%%%%%%%%%%%%%%%%%%%%%%%%%%%%%%%%%%%%%%%%%%%%%%%%%%%%%%%%%%%%%%%%%%%%%%%%
%%% 画像ファイルの include path
\graphicspath{{fig/}} %% グラフィック用


%%%%%%%%%%%%%%%%%%%%%%%%%%%%%%%%%%%%%%%%%%%%%%%%%%%%%%%%%%%%%%%%%%%%%%%%%%%%%%
%%% マクロ一覧 (必要なマクロをこの部分に記述)

\newcommand{\bA}{\mathbf{A}}
\newcommand{\bB}{\mathbf{B}}


%%%%%%%%%%%%%%%%%%%%%%%%%%%%%%%%%%%%%%%%%%%%%%%%%%%%%%%%%%%%%%%%%%%%%%%%%%%%%%
%%% タイトル,著者,所属,概要

% 日本語タイトル
\jtitle{
\workname{}:\colorname{}に用いられる漢字と\\
\colorname{}が表す色の関係性に関する可視化手法の提案と考察
}

% 英語タイトル
\etitle{
\worknameen{}
}

% 日本語著者
% 所属参照は好みに応じて \dagger などを用いてもよい.
\jauthor{
遠藤勝也\(^{1)}\){\small (非会員)}
% ~~~ % 氏名の間隔はチルダ記号や hspace 等で適宜調節のこと.
% 科学次郎\(^{2)}\){\small (正会員)}
}

% 英語著者
\eauthor{
Katsuya Endoh\(^{1)}\)
% ~~~
% Jiro Kagaku\(^{2)}\)
}

% 日本語所属
\jaffiliation{
1) 株式会社スタジオ・アルカナ
% ~~~
% 2) 芸術科学大学芸術科学部
}

% 英語所属
\eaffiliation{
1) Studio Arcana co.,Ltd.\\
% 2) Department of Art and Science, The University for Art and Science
}

% 連絡先電子メールアドレス(省略可)
% (スパム対策は著者自身の判断によって措置すること.
% このサンプルでは「@」を2バイト文字にすることで対応してある.)
\email{
endkty0509@gmail.com
}

% 日本語概要
\jabstract{
色を表記する方法として,\colorname{}を用いることがある.
たとえば,深緋という\colorname{}が,
「深」と「緋」の二つの漢字から構成されていることに注目すると,
日本語話者には「深」という漢字からは暗さや濃さ,
「緋」という漢字からは赤に近い色のような感覚が想起されると考えられる.
そこで,本研究では,\colorname{}を漢字に分解することで,
\colorname{}に用いられる漢字の関係性を可視化した作品\workname{}を制作し,
漢字が「\colorname{}から想起させる色」に与えている影響について考察する.
}

%%%%%%%%%%%%%%%%%%%%%%%%%%%%%%%%%%%%%%%%%%%%%%%%%%%%%%%%%%%%%%%%%%%%%%%%%%%%%%
% ここより論文本体

\begin{document}
\maketitle
\thispagestyle{myheadings}

\mysection{はじめに}

色を表記する方法として,\colorname{}を用いることがある.
たとえばRGB値で(134, 71, 63)と表される色は,
深緋(こきあけ)という\colorname{}で表されることがある.
上記の値から暗く濃い赤色だということは想像することは,
色を数値で扱う習慣がなければ難しいと考えられる.
しかし,深緋という\colorname{}が,
「深」と「緋」の二つの漢字から構成されていることに注目すると,
日本語話者には「深」という漢字からは暗さや濃さ,
「緋」という漢字からは赤に近い色のような感覚が想起されると考えられる.
そこで,本研究では,\colorname{}を漢字に分解することで,
\colorname{}に用いられる漢字の関係性を可視化した作品\workname{}を制作し,
漢字が「\colorname{}から想起させる色」に与えている影響について考察する.

\mysection{先行研究および関連作品}

色彩語における漢字の果たす役割として,
木村ら\cite{Kimura1998}は「赤」や「紅」,「朱」などの漢字から想起される色は,
対照可能な環境下において異なる色を想起させると述べている.

また,日本の伝統色を可視化した作品として,
NIPPON COLORS\cite{NipponColors}があげられる.
NIPPON COLORSは日本の伝統色を扱った色見本サイトだが,
色見本の表示方法として,各色をマンセルの色立体を使って可視化している.

上記のように,色彩語における漢字の果たす役割に関する研究や,
日本の伝統色を可視化した作品の制作は多く行われているが,
色彩語を構成する漢字の関係性を可視化した作品は少ない.

\section{「\workname{}」について}

「\workname{}」はNo.1からNo.3の三つの可視化作品から構成される.
これらは\colorname{}とそれに紐づく色彩のデータベースを使い制作されている.
以下の節にてそれぞれの作品について述べる.

\mysubsection{No.1について}

「\workname{} No.1」は,
\colorname{}を漢字に分解し,
それぞれの漢字が用いられる色彩の平均色を計算し,
3次元空間内に配置した可視化作品である.

\begin{figure}[htbp]
  \begin{center}
    \includegraphics[width=7cm,clip]{fig/kanji-color-space.eps}
    \caption{No.1}
    \label{no1}
  \end{center}
\end{figure}

\mysubsection{No.2について}

「\workname{} No.2」は,
\colorname{}をノード,
同じ漢字を使用している関係性をエッジとして,
グラフ構造を可視化した作品である.
ノードは力学モデルによって配置されるため,
同じ漢字を使用している\colorname{}のクラスタを観察することが可能である.

\begin{figure}[htbp]
  \begin{center}
    \includegraphics[width=7cm,clip]{fig/kanji-color-graph.eps}
    \caption{No.2}
    \label{no2}
  \end{center}
\end{figure}

\mysubsection{No.3について}

「\workname{} No.3」は,
「\workname{} No.2」と同じ関係性を用いてグラフ構造を可視化した作品である.
ノードは\colorname{}の色相によって円形に配置されていているため,
\colorname{}の色相環上における関係性を観察することが可能である.

\begin{figure}[htbp]
  \begin{center}
    \includegraphics[width=7cm,clip]{fig/kanji-circle-color-graph.eps}
    \caption{No.3}
    \label{no3}
  \end{center}
\end{figure}

% \workname{}は3D空間に配置された,
% \colorname{}に使われている漢字(以降,\colorkanji{}とする)と,
% \colorkanji{}が表す色(以降,\recallcolor{}とする)を鑑賞可能な可視化作品である.
% 
% 本作品における,基本となる色データベースは,
% XX種類の色とXX種類の\colorname{}が登録されていて,
% 色と\colorname{}は一対多の関係にある.
% 
% \colorname{}に使われている\colorkanji{}の\recallcolor{}は,
% それぞれの\colorkanji{}が使われている色の平均色を割り当てることで表現した.
% また,「色」という漢字は,
% 「その単語が色を示している」という意味のみを表現していると考えられる.
% そのため,本作品では\recallcolor{}に影響を与えないものとして考える.
% よって,「肌色」のような\colorname{}は,
% 「肌」という\colorkanji{}からのみ構成されている考えることとする.
% 
% \workname{}では,\colorkanji{}の3Dモデルを,
% 上記の手法によって計算した\recallcolor{}でレンダリングし,
% HSV値を空間上にマッピングして配置した.
% また,\recallcolor{}は色の範囲が広いため,
% 背景を単色にした際に,
% 同色系の\colorkanji{}の鑑賞が難しくなる.
% よって,本作品は漢字をテーマとしていることから,
% 背景には和風な幾何学模様を表示した.
% カメラの視点は,マウスドラッグとスクロールにより変更可能である.
% また,マウスクリックもしくは,
% セレクトボックスから注目したい漢字を選択することで,
% 選択した漢字を中心とした鑑賞が可能である.
% 図\ref{works1}と図\ref{works2}に作品が動作している様子を示す.

% TODO: 参考になり得るデータを示す
% 深緋を例として考えると,深緋が示す色はRGB値で(134, 71, 63)である.
% また,深紫が示す色はRGB値で(74, 34, 93)である.

% 緋が示す色はRGB値で(204, 84, 58)である.
% 紫が示す色はRGB値で(89, 44, 99)である.

% また,本作品では漢字は単色で表示されるので,
% 背景が単色だと観察しにくい色の漢字が出てきてしまう可能性が考えられるので,
% 背景は和柄の幾何学模様をアニメーションさせた.
% TODO: アニメーションさせた

\mysection{考察}

「\workname{} No.1」からは
「海」という漢字に注目した時に,
青に近い\recallcolor{}を想起させられると考えられるが,
平均色は(119, 90, 58)と茶色に近い色となった.
この理由として,「海」が用いられている\colorname{}を観察した結果,
6語の\colorname{}の内,3語が「海老」として用いられていたからだと考えられる.
この結果から,単語も考慮する必要があると考えられる.

また,「\workname{} No.2」からは,
「赤」や「紅」,「朱」が同一ではない隣接するクラスタを作っていることが観察できた.
また,観察した結果から,これらのクラスタがわずかに異なる色の系統を表していると考えられる.

最後に,「\workname{} No.3」からは,
茶色周辺には多くのエッジが密集していることが観察できたが,
逆に青周辺はエッジがまばらな点から,
茶色を表す漢字は少なく,
青を表す漢字は豊富だと考えられる.

\mysection{おわりに}

本稿では,可視化作品\workname{}を制作し,
漢字が\recallcolor{}に与える影響を考察した.
今後の展望として,
\colorname{}から想起される色のデータをアンケートから収集することで,
ユーザ特性によって生じる,想起される色の違いを可視化することが可能だと考えられる.

%%%%%%%%%%%%%%%%%%%%%%%%%%%%%%%%%%%%%%%%%%%%%%%%%%%%%%%%%%%%%%%%%%%%%%%%%%%%%%
%%% 参考文献

\bibliography{bibtex_samp}
\bibliographystyle{junsrt}

\end{document}
