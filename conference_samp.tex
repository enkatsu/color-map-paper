% \documentclass[a4paper,twocolumn]{jsarticle}
\documentclass[a4j,twocolumn]{ujarticle} % 'jsarticle' が使えない場合はこちらを利用

% 上記 documentclass のオプションは自由に追加してもよい.

\newcommand{\visualizationtitle}{「作品名」}
\newcommand{\colorkanji}{色用漢字}
\newcommand{\recallcolor}{想起色}

%%%%%%%%%%%%%%%%%%%%%%%%%%%%%%%%%%%%%%%%%%%%%%%%%%%%%%%%%%%%%%%%%%%%%%%%%%%%%%
%%% ページ設定 (この項目は論文著者は編集しないこと.)

% 芸術科学会学術会議用スタイルパッケージ
\usepackage{artsci-conf-j}

% 開始ページ数設定
\setcounter{page}{1}

% ヘッダスタイル
\markright{\footnotesize{
\textbf{NICOGRAPH 20xx, pp. aaa -- bbb}
}}
\pagestyle{myheadings}

%%%%%%%%%%%%%%%%%%%%%%%%%%%%%%%%%%%%%%%%%%%%%%%%%%%%%%%%%%%%%%%%%%%%%%%%%%%%%%
%%% パッケージ一覧 (必要なパッケージを任意に追加してよい)

\usepackage{amsmath, amssymb}	% AMS-LaTeX
\usepackage[dvipdfmx]{graphicx}	% 「graphics」パッケージに変更してもよい.
\usepackage{float}		% 図表が記述位置から飛ばないためのパッケージ
\usepackage{url}		% URL 表記用パッケージ

%%%%%%%%%%%%%%%%%%%%%%%%%%%%%%%%%%%%%%%%%%%%%%%%%%%%%%%%%%%%%%%%%%%%%%%%%%%%%%
%%% 画像ファイルの include path
\graphicspath{{fig/}} %% グラフィック用


%%%%%%%%%%%%%%%%%%%%%%%%%%%%%%%%%%%%%%%%%%%%%%%%%%%%%%%%%%%%%%%%%%%%%%%%%%%%%%
%%% マクロ一覧 (必要なマクロをこの部分に記述)

\newcommand{\bA}{\mathbf{A}}
\newcommand{\bB}{\mathbf{B}}


%%%%%%%%%%%%%%%%%%%%%%%%%%%%%%%%%%%%%%%%%%%%%%%%%%%%%%%%%%%%%%%%%%%%%%%%%%%%%%
%%% タイトル,著者,所属,概要

% 日本語タイトル
\jtitle{
\visualizationtitle{}:\\
色名に用いられる漢字から想起される\\
色空間内における漢字の可視化手法の提案
}

% 英語タイトル
\etitle{
Using kanji for color name ~~ visualization
}

% 日本語著者
% 所属参照は好みに応じて \dagger などを用いてもよい.
\jauthor{
遠藤勝也\(^{1)}\){\small (非会員)}
% ~~~ % 氏名の間隔はチルダ記号や hspace 等で適宜調節のこと.
% 科学次郎\(^{2)}\){\small (正会員)}
}

% 英語著者
\eauthor{
Katsuya Endoh\(^{1)}\)
% ~~~
% Jiro Kagaku\(^{2)}\)
}

% 日本語所属
\jaffiliation{
1) 株式会社スタジオ・アルカナ
% ~~~
% 2) 芸術科学大学芸術科学部
}

% 英語所属
\eaffiliation{
1) Studio Arcana co.,Ltd.\\
% 2) Department of Art and Science, The University for Art and Science
}

% 連絡先電子メールアドレス(省略可)
% (スパム対策は著者自身の判断によって措置すること.
% このサンプルでは「@」を2バイト文字にすることで対応してある.)
\email{
endkty0509@gmail.com
}

% 日本語概要
\jabstract{
色を表記する方法として,色名を用いることがある.
たとえばRGB値で(134, 71, 63)と表される色は,
深緋(こきあけ)という色名で呼ばれることがある.
上記の値から濃い赤色だということは想像できるが,
色を数値で扱う習慣がなければ,
数値から色を想像することは難しいと考えられる.
しかし,深緋という色名は,「深」と「緋」の二つの漢字から構成されていることに注目すると,
日本語話者には「深」という漢字からは暗さや濃さ,
「緋」という漢字からは赤に近い色のような感覚が想起されると考えられる.
そこで本研究では,色名を漢字に分解し,
それらの漢字が用いられる色を元に,
漢字を色空間内に表現した可視化作品\visualizationtitle{}を制作し,
漢字が色名から想起させる色に与えている影響について考察する.
}

%%%%%%%%%%%%%%%%%%%%%%%%%%%%%%%%%%%%%%%%%%%%%%%%%%%%%%%%%%%%%%%%%%%%%%%%%%%%%%
% ここより論文本体

\begin{document}
\maketitle
\thispagestyle{myheadings}

\section{はじめに}

色を表記する方法として,色名を用いることがある.
たとえばRGB値で(134, 71, 63)と表される色は,
深緋(こきあけ)という色名で呼ばれることがある.
上記の値から濃い赤色だということは想像できるが,
色を数値で扱う習慣がなければ,
数値から色を想像することは難しいと考えられる.
しかし,深緋という色名は,「深」と「緋」の二つの漢字から構成されていることに注目すると,
日本語話者には「深」という漢字からは暗さや濃さ,
「緋」という漢字からは赤に近い色のような感覚が想起されると考えられる.
そこで本研究では,色名を漢字に分解し,
それらの漢字が用いられる色を元に,
漢字を色空間内に表現した可視化作品\visualizationtitle{}を制作し,
漢字が色名から想起させる色に与えている影響について考察する.

\section{先行研究および関連作品}

文字を見ると色が感じられる現象は,色字共感覚と呼ばれる.
色字共感覚に関する研究として,宇野ら\cite{Uno2018}の〜〜〜があげられる.

日本の伝統色を可視化した作品として,
NIPPON COLORS\cite{NipponColors}があげられる.
NIPPON COLORSは日本の伝統色を扱った色見本サイトだが,
色見本の表示方法として,各色をマンセルの色立体を使って可視化している.

\section{\visualizationtitle{}について}

\subsection{技術説明}

本作品では,色名に使われている漢字(以降,\colorkanji{}とする)に,
それぞれ使われている色の平均色を割り当てることで,
\colorkanji{}が想起される色(以降,\recallcolor{}とする)に
どのような影響を与えているのかを考察した.
また,本研究では「色」という漢字は,
「その単語が色を示している」という意味のみを表現していると考えられるため,
\recallcolor{}に影響を与えないものとして考える.
よって,「肌色」のような色名は,
「肌」という\colorkanji{}からのみ構成されている考えることとする.
深緋を例として考えると,深緋が示す色はRGB値で(134, 71, 63)である.
また,深紫が示す色はRGB値で(74, 34, 93)である.

緋が示す色はRGB値で(204, 84, 58)である.
紫が示す色はRGB値で(89, 44, 99)である.


基本となる色データベースは,
XX個の色とXX個の色名が登録されて,色と色名は一対多の関係にある.

\section{おわりに}

本稿では,漢字が色イメージに与える色空間内におけるベクトルを推測し,その結果を考察した.

%%%%%%%%%%%%%%%%%%%%%%%%%%%%%%%%%%%%%%%%%%%%%%%%%%%%%%%%%%%%%%%%%%%%%%%%%%%%%%
%%% 参考文献

\bibliography{bibtex_samp}
\bibliographystyle{junsrt}

\end{document}
